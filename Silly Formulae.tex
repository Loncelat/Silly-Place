\pdfcompresslevel=0
\documentclass{article}

\usepackage{amsmath}
\usepackage{amsfonts}
\usepackage{parskip}
\usepackage[a4paper]{geometry}
\usepackage[symbol]{footmisc}

\newcommand{\floor}[1]{\left\lfloor #1 \right\rfloor}
\newcommand{\dif}[1]{\,\mathrm{d}#1}
\newcommand{\intervaloo}[2]{\left[#1, #2\right]}
\newcommand{\Mod}{\,\mathrm{mod}\,}

\begin{document}

\title{Silly Formulae}
\author{Martijn Leus}
\date{NULL}

\maketitle

\section{Patchwork}

Let $a$ be the size (in pixels) of each patch, with $a\in\mathbb{N}\setminus\{0\}$.\\
Let $b$ be the width of each line of the grid, with $b\in\mathbb{N}$.

The leftmost pixel $p_l=L(n)$ of the $n^{th}$ patch is then given by
$$L(n)=an + bn + b$$

Similarly, the rightmost pixel  $p_r=R(n)$ of the $n^{th}$ patch is
\begin{align*}
	R(n) &= L(n) + a - 1 \\
	&= an + bn + a + b - 1
\end{align*}

Formulae for the left- and rightmost pixels of the next patch can be derived from these formulae:
\begin{align*}
	L(n+1) &=a(n+1) + b(n+1) + b \\
	&= an + bn + b + a + b \\
	&= L(n) + a + b
\end{align*}
\begin{align*}
	R(n+1) &=L(n+1) + a - 1 \\
	&= L(n) + a + b + a - 1 \\
	&= R(n) + a + b
\end{align*}

Clearly, the difference between the leftmost (and rightmost) pixels of two consecutive patches is always $a + b$.

Assuming that values are always floored, the function $n=L^{-1}(p_l)$ is
$$L^{-1}(p_l)={\frac{p_l-b}{a+b}}$$
This also gives us a formula for the the patch which the next leftmost pixel of a patch belongs to:
$$L^{-1}(p_l + (a + b))=\frac{p_l-b+(a+b)}{a+b}=\frac{p_l-b}{a+b}+\frac{a+b}{a+b}=\frac{p_l-b}{a+b}+1$$

Say, we want to know if a given pixel is part of a patch, and not of a border. The first step in doing this is figuring out at which offset $s$ from the start of the border-patch pair the pixel is. This can be calculated by $s=p\Mod(a + b)$. This function maps $p$ onto the closed interval $\intervaloo{0}{a+b-1}$.

Notice that the first $b$ items in this interval belong to a border, thus the border is the closed interval $\intervaloo{0}{b-1}$. In order to find out if $p$ maps onto a patch, we then need to check if $s$ falls outside of this interval. 
To do this, we evaluate $s> b-1$\footnote[2]{$s < 0$ isn't evaluated because $s$ can never be less than 0.}.
This is equivalent to $s\geq b$.
Filling in the previously defined formula for $s$, we arrive at $$p\Mod(a + b) \geq b$$

\section{$ln$tegrals}
$$\int ln^n(x)\dif{x}=
\begin{cases}
x(ln^n(x)-\frac{n!}{(n-1)!}ln^{n-1}+\frac{n!}{(n-2)!}ln^{n-2}-...+n!\cdot ln(x)-n!)+C & \text{if } n=even \\ 
x(ln^n(x)-\frac{n!}{(n-1)!}ln^{n-1}+\frac{n!}{(n-2)!}ln^{n-2}-...-n!\cdot ln(x)+n!)+C & \text{if } n=odd
\end{cases}$$

$$\int ln^n(x)\dif{x}=xn!\sum_{k=0}^{n}\frac{(-1)^{n-k}ln(x)^k}{k!}+C$$


\end{document}